

%% Classe de documento
\documentclass[%% Opções (^ = padrão; > = para pacotes):
  a4paper,%% Tamanho de papel: a4paper, letterpaper (^), etc.
  12pt,%% Tamanho de fonte: 10pt (^), 11pt, 12pt, etc.
  fleqn,%% Alinhamento de equações à esquerda (centralizado por padrão)
%   draft,%% Aparência de documento (>): draft ou final (^)
  english,%% Idioma secundário (penúltimo) (>)
  brazilian,%% Idioma primário (último) (>)
]{article}

%% Pacotes utilizados
\usepackage[%% Opções (^ = padrão; ¹ = Leiaute A; ² = Leiaute B):
%   Layout    = A,%% Leiaute (margens): A (esquerda menor + recuo) ou B (iguais) (^)
%   Font      = Calibri,%% Fonte principal: Arial (²), Calibri (¹), Times ou LaTeX
%   RuleWidth = Text,%% Largura de linha (cabeçalho e rodapé): Zero (²), Text ou Paper (¹)
%   AffilIcon = Off,%% Ícones de e-mail, Lattes, ORCID e Instituição: On (^) ou Off
%   DocLic    = None,%% Licença do documento: CC (Creative Commons) (^) ou None
%   CCLic     = BY-NC-ND,%% Licença CC: BY (^), BY-SA, BY-ND, BY-NC, BY-NC-SA ou BY-NC-ND
%   Link      = TextColor,%% Cor de hyperlinks: DarkBlue (^) ou TextColor
%   PageNum   = Off,%% Numeração de páginas: On (^) ou Off
%   SectNum   = Off,%% Numeração de seções (\section): On (²) ou Off (¹)
%   SectUnnum = Center,%% Alinhamento de seções (\section*): Center (¹) ou Left (²)
%   Caption   = Left,%% Alinhamento de legendas: Center (^) ou Left
%   Source    = Left,%% Alinhamento de fonte (referência): Center (^) ou Left
%   ABNTCit   = NSB,%% Citação ABNT: AAY (NOME, ANO) (^), NRB (1) ou NSB [1]
%   BibDOI    = Icon,%% Ícone de DOI em referências: Icon ou Name (^)
%   BibURL    = Icon,%% Ícone de URL em referências: Icon ou URL (^)
]{utfpr-article}

\usepackage[hyperlink]{qrcode}
\usepackage{environ}
\usepackage{tabularray}
\usepackage[fleqn]{nccmath}
\usepackage{bm}
\relpenalty=9999
\binoppenalty=9999
\NewEnviron{ceqnalign*}
    {\vspace{3mm}
    \begin{ceqn}
        \begin{align*}
            \BODY
        \end{align*}
\end{ceqn}}
\NewEnviron{ceqnalign}[1]
{\vspace{3mm}
	\begin{ceqn}
		\begin{align*}
			\BODY \tag{#1}
		\end{align*}
\end{ceqn}}
\usepackage{catchfilebetweentags}
\DeclareMathOperator{\Ima}{Im}
\newcommand{\Laplace}{\mathscr{L}}
\newcommand{\bN}{\mathbb N}
\newcommand{\inN}{\in \bN}
\newcommand{\dotline}{\\.\dotfill.\\}
\newcommand{\sqn}[1]{(#1_n)_{n \in \bN}}
\newcommand{\fmm}[3]{\{#1_#2\}_{#2 \in #3}}
\newcommand{\seq}[3]{(#1_#2)_{#2 \in #3}}
\newcommand{\tozero}[1]{#1 \rightarrow 0}
\newcommand{\toinf}[1]{#1 \rightarrow \infty}
\newcommand{\limsq}[3]{\lim_{#1 \rightarrow #2} #3_#1}
\newcommand{\limif}[2]{\lim_{#1 \rightarrow \infty} #2}
\newcommand{\limzr}[2]{\lim_{#1 \rightarrow 0} #2}
\newcommand{\bC}{\mathbb C}
\newcommand{\bZ}{\mathbb Z}
\newcommand{\bR}{\mathbb R}
\newcommand{\bQ}{\mathbb Q}
\newcommand{\itr}[1]{\text{int}(#1)}
\newcommand{\parth}[1]{\left(#1\right)}
\newcommand{\chavs}[1]{\left\{#1\right\}}
\newcommand{\mdl}[1]{\left|#1\right|}
\newcommand{\xssy}[2]{\left(#1 - #2, #1 + #2\right)}
\newcommand{\floor}[1]{\left\lfloor#1\right\rfloor}
\newcommand{\ceil}[1]{\left\lceil#1\right\rceil}
\newcommand{\asbeps}[2]{|#1 - #2| < \epsilon}
\newcommand{\iffc}[1]{\overset{#1}{\iff}}
\newcommand{\impliesc}[1]{\overset{#1}{\implies}}
\newcommand{\eqc}[1]{\overset{#1}{=}}
\newcommand{\sen}{\mathrm{sen}}

%%%% Autor(es) (de 1 a 8, Leiaute A; de 1 a 16, Leiaute B): {Número}; {Dados}
 \Author{1}{%% Bolsista ou voluntário(a) principal (primeiro)
   Name        = {Fabio Zhao Yuan Wang},%
   Email       = {fabioyuan@gmail.com},%
 }
% \Author{2}{%% Colaborador(a) (segundo ao penúltimo)
%   Name        = {Nome Completo do{(a)} Autor{(a)}-A2},%
%   Email       = {author2@domain},%
%   Lattes      = {0000000000000002},%% Opcional
%   ORCID       = {0000-0000-0000-0002},%% Opcional (CHKTEX 8)
%   Affiliation = {Instituição (Nome por Extenso), Cidade, Estado, País},%
%   Role        = {Discente de Nome do Curso},%% Leiaute B
% }
% \Author{3}{%% Orientador(a) principal (último)
%   Name        = {Nome Completo do{(a)} Autor{(a)}-A3},%
%   Email       = {author3@domain},%
%   Lattes      = {0000000000000003},%% Opcional
%   ORCID       = {0000-0000-0000-0003},%% Opcional (CHKTEX 8)
%   Affiliation = {\UTFPRName, Cidade, Paraná, Brasil},%
%   Role        = {Docente do Nome do Departamento ou Programa},%% Leiaute B
% }
%%%% Digital Object Identifier (DOI): {Name}
% \DOIName{10.1000/xyz123}
%%%% Datas: [Ano] (atual por padrão; opcional); {Recebido}; {Aprovado}
% \Dates[2023]{DD mmm. \YearNum}{DD mmm. \YearNum}
%%%% Evento (SEI, SICITE, etc.): {Dados}
% \EventInfo{%
%   Name    = {Nome Completo do Evento},%
%   Acronym = {EVNT \YearNum},%
%   Date    = {DD a DD de Mmmmmm de \YearNum},%
%   Local   = {Cidade, Paraná, Brasil},%
%   Logo    = {example-image},%% Arquivo de imagem em ./Logos/
%   Header  = {example-image},%% Arquivo de imagem em ./Headers/
%   URL     = {https://www.example.com/},%
% }
%%%% Instituição (se nenhum evento): {Dados}
 \InstitutionInfo{%
   Campus     = {Curitiba},%
   Department = {Departamento de Eletrônica},%
 }


%% Arquivo(s) de referências
\addbibresource{utfpr-article.bib}

%% Início do documento
\begin{document}
\begin{enumerate}[wide, labelwidth=!, labelindent=0pt]
  \item \textbf{Como vimos, existe uma função injetora $\dfunc s \bN \bN$, tal que $s(n) = n+1$, em que 
    $s(n)$ chama-se \textit{sucessor} de $n$. A partir disso, podemos definir a operação \textit{soma} em 
    $\bN$, que satisfaz a seguinte lei de recursão: }\\
  \begin{itemize}
    \item \textbf{Para todo $n \inN$, temos que $n+1 = s(n)$;}
    \item \textbf{Para cada $m, n \inN$, temos que $n + s(m) = s(n+m)$.}
  \end{itemize}
  \vspace{3mm}
  \textbf{Também podemos definir a operação de produto em $\bN$, que satisfaz a seguinte lei de recursão:}\\
  \begin{itemize}
    \item \textbf{Para todo $n\inN$, temos que $n\cdot 1 = n$;}
    \item \textbf{Para cada $m, n \inN$, temos que $n\cdot s(m) = nm+n$.}
  \end{itemize}
    \vspace{3mm}
    \textbf{Mostre que valem as seguintes propriedades:}
  \begin{enumerate}[label=\alph*)]
      \item \textbf{(Comutatividade da soma) Para cada $m, n \inN$, temos que $m+n = n+m$.}\\
        \ExecuteMetaData[Section/sol.tex]{ex1a}\\
      \item \textbf{(Lei do cancelamento da soma) Para cada $m, n, p \inN$, se $m+p = n+ p$, então $m=n$.}
        \ExecuteMetaData[Section/sol.tex]{ex1b}\\

      \item \textbf{(Comutatividade do produto) Para cada $m, n \inN$, temos que $mn = nm$.}\\
        \ExecuteMetaData[Section/sol.tex]{ex1c}\\
      \item \textbf{(Distributividade) Para cada $m,n,p \inN$, temos que $m(n+p) = mn + mp$.}\\
        \ExecuteMetaData[Section/sol.tex]{ex1d}\\
      
      \item \textbf{(Lei do cancelamento do produto) Para cada $m,n,p \inN$, se $mp=np$, então m=n.}
        \ExecuteMetaData[Section/sol.tex]{ex1e}\\
      
      \item \textbf{(Unicidade da identidade do produto) Para todo $n \inN$, se $mn = n$, então $m = 1$}\\
        \ExecuteMetaData[Section/sol.tex]{ex1f}\\
  \end{enumerate} 
	  \vspace{3mm}

  \item \textbf{Sejam $m, n, p \inN$. Mostre que}

  \begin{enumerate}[label=\alph*)]
    \item \textbf{Se $m \le n$ e $n < p$, então $m < p$.}\\
    \ExecuteMetaData[Section/sol.tex]{ex2a}\\
    \item \textbf{Se $m < n$ e $n \le p$, então $m < p$.}\\
    \ExecuteMetaData[Section/sol.tex]{ex2b}\\
    \item \textbf{Se $m \le n$ e $n \le p$, então $m \le p$.}\\
    \ExecuteMetaData[Section/sol.tex]{ex2c}\\
  \end{enumerate}
  
\end{enumerate}


\end{document}
