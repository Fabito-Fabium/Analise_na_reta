%<*ex1a>
Antes, considere a seguinte proposição:
\dotline
\textbf{Proposição:} Seja $m \inN$, então $m+1 = 1+ m$.\\
\textbf{Demonstração da proposição:} Considere $S \subset \bN$ tal que, $S = \{m \inN 
; m+1=1+m\}$. Note que $1 \in S$, já que $(1) + 1 = 2 = 1 + (1)$; Vejamos agora que $
s(S) \subset S$. Seja $m \in S$, então:
\begin{ceqnalign*}
  s(m)+1 &= (m+1) + 1, \quad &&\text{(Definição da função sucessor)}\\
  &= (1+m) + 1, \quad &&\text{(Visto que $m \in S$)} \\
  &= 1 + (m + 1), \quad &&\text{(Associatividade da soma em $\bN$)}\\
  &= 1 + s(m). \quad &&\text{(Definição da função sucessor)}
\end{ceqnalign*}

Deste modo, já que $1 \in S$ e $s(S) \in S$, pelo Princípio da Indução
, temos que $S = \bN$, como queríamos. $\square$
\dotline
\textbf{Dem:} Considere $S \subset \bN$ tal que $S = \{m \inN; m+n = n+m, n\inN\}$
. Da proposição anterior, temos que $1 \in S$, visto que $1 + n = n + 1$. Vejamos que $
s(S) \subset S$. Tomando $m \in S$,\\
\begin{ceqnalign*}
  s(m)+n &= (m+1) + n, \quad &&\text{(Definição da função sucessor)}\\
  &= m + (1+n), \quad &&\text{(Associatividade da soma em $\bN$)} \\
  &= m + (n+1), \quad &&\text{(Proposição anterior)}\\
  &= (m+n) + 1, \quad &&\text{(Associatividade da soma em $\bN$)}\\
  &= (n + m) + 1, \quad &&\text{(Visto que $m \in S$)}\\
  &= n + (m+1), \quad &&\text{(Associatividade da soma em $\bN$)}\\
  &= n + s(m). \quad&&\text{(Definição da função sucessor)}
\end{ceqnalign*}

Visto que $1 \in S$ e $s(S) \subset S$, pelo Princípio da Indução, temos $S=\bN$
, como queríamos. $\square$
%</ex1a>

%<*ex1b>
\textbf{Dem: } Seja $S \subset \bN$ tal que $S=\{p \inN ; (m+p=n+p) \implies m=n, \quad 
m,n \in \bN\}$. Ora, $1 \in S$, já que, se $m+1 = n+1$, então $s(m) = s(n)$, e
, como $s$ é injetora por definição, então $m=n$. Vejamos que $s(S) \subset S$. Toman
do $p \in S$ e $m+s(p) = n + s(p)$,
\begin{ceqnalign*}
  m+s(p)&=n+s(p),\\
  m+(p+1) &=n+(p+1), \quad &&\text{(Definição da função sucessor)}\\
  (m+p) + 1 &= (n+p) + 1, \quad &&\text{(Associatividade da soma em $\bN$)}\\
  s(m+p)&=s(n+p), \quad &&\text{(Definição da função sucessor)}\\
  m+p &=n+p, \quad &&\text{(Injetividade da função sucessor)}\\
  m &= n. \quad &&\text{(Já que $p \in S$)}
\end{ceqnalign*}

Deste modo, como $1 \in S$ e $s(S) \subset S$, pelo Princípio da Indução, temos $S 
= \bN$. $\square$
%</ex1b>

%<*ex1c>
Antes, considere as seguintes proposições:
\dotline
\textbf{Proposição 1:} Seja $n \inN$, $n\cdot 1 = 1 \cdot n$.\\
\textbf{Demonstração da proposição:} Seja $T \subset \bN$ tal que $T = \{n\inN; n\cdot 1
=1\cdot n\}$, queremos mostrar que $T = \bN$. Ora, $1 \in T$, já que pela defini
ção do produto em $\bN$, $(1)\cdot 1 = 1 = 1 \cdot(1)$. Vejamos que $s(T) \subset T$
. Tomando $n \in T$, temos que:\\
\begin{ceqnalign*}
  s(n)\cdot 1 &= s(n), \quad&&\text{(Definição do produto em $\bN$)}\\
  &=n + 1, \quad&&\text{(Definição da função sucessor)}\\
  &=1\cdot n + 1, \quad&&\text{(Da definição do produto e sabendo que $n \in T$
  , $n = n\cdot 1 = 1 \cdot n$)}\\
  &= 1\cdot s(n). \quad &&\text{(Definição do produto em $\bN$, $m\cdot s(n) = mn+m$)}
\end{ceqnalign*}
Portanto, como $1 \in T$ e $s(T) \subset T$, segue do Princípio da Indução que $T=\bN$
. $\square$
\dotline
\textbf{Proposição 2:} (Distributiva comutada) Para cada $m, n, p \inN$, temos que 
\\
$$(n+p)m = nm+pm.$$
\textbf{Demonstração da proposição:} Considere $R=\{m \inN; (n+p)m = nm + pm, 
\quad n,p\inN\}$. Note que $1 \in R$, visto que, da definição de produto, $
(n+p)\cdot 1 = n + p = n\cdot 1 + p\cdot 1$. Vejamos que $s(R) \subset R$. Assumindo $
m \in R$, temos que:
\begin{ceqnalign*}
  (n+p)s(m) &= (n+p)m + (n+p), \quad &&\text{(Definição do produto em $\bN$)}\\
  &= (nm + pm) + (n+p), \quad &&\text{(Visto que $m \in R$, $(n+p)m = nm + pm$)}\\
  &= nm + (pm + (p+n)), \quad &&\text{(Associatividade e comutatividade da soma em $
  \bN$)}\\
  &= nm + (n + (pm + p)), \quad &&\text{(Associatividade e comutatividade da soma em $
  \bN$)}\\
  &= (nm + n) + (pm + p), \quad &&\text{(Associatividade da soma em $\bN$)}\\
  &= n\cdot s(m) + p \cdot s(m). \quad &&\text{(Definição do produto em $\bN$)}
\end{ceqnalign*}
Sendo assim, já que $1 \in R$ e $s(R) \subset R$, segue do Princípio da Indu
ção que $R = \bN$. $\square$
\dotline
\textbf{Dem: } Seja $S = \{n \inN | m\cdot n = n \cdot m, \quad m \inN\}$. Veja
mos que $S = \bN$. Da proposição 1, temos que $1 \in S$. Para verificar que $s(S) 
\subset S$, suponhamos que $n \in S$, tendo assim:
\begin{ceqnalign*}
  m\cdot s(n) &= mn + m, \quad &&\text{(Definição do produto em $\bN$)}\\
    &= nm + m, \quad &&\text{(Visto que $n \in S$, $mn = nm$)}\\
    &= nm + 1m, \quad &&\text{(Da proposição 1, $m=m\cdot 1 = 1 \cdot m$)}\\
    &= (n + 1)m, \quad &&\text{(Distributiva comutada)}\\
    &= s(n) \cdot m. \quad &&\text{(Definição da função sucessor)}
\end{ceqnalign*}
Deste modo, como $1 \in S$ e $s(S) \subset S$, do Princípio da Indução, temos q
ue $S = \bN$, como queríamos. $\square$
%</ex1c>


%<*ex1d>
\textbf{Dem: } Ora, da \textit{distributiva comutada}, demonstrada no item anterior
, temos que, para cada $m,n,p \inN$, $(n+p)m = nm +pm$. Portanto,
\begin{ceqnalign*}
  m(n+p) &= (n + p)m, \quad &&\text{(Comutatividade do produto em $\bN$)}\\
  &= nm + pm, \quad &&\text{(Distributiva comutada)}\\
  &= mn + mp. \quad &&\text{(Comutatividade do produto em $\bN$)}
\end{ceqnalign*}
Com isto, temos o que queríamos. $\square$
%</ex1d>


%<*ex1e>
Antes, mostraremos a unicidade da identidade do produto em $\bN$.
\dotline
\textbf{Proposição: } (Unicidade da identidade do produto) Para todo $n \in 
N$, se $mn = m$, então $m = 1$.\\
\textbf{Demonstração da proposição: } A fim de absurdo, suponhamos que $m \neq 1$
, isto é, existe $k \inN$ tal que $m = s(k)$, ou seja $mn = n$ pode ser reescrito como:
\begin{ceqnalign*}
  n &= s(k) \cdot n,\\
  &= n\cdot s(k), \quad &&\text{(Comutatividade do produto em $\bN$)}\\
  &= nk + n. \quad &&\text{(Definição do produto em $\bN$)} \tag{$\star$}
\end{ceqnalign*}

Porém, como visto em aula, para cada $x,y \inN$, $x\neq x + y$, sendo assim, ($\star
$) é um absurdo. Deste modo, temos o que queríamos. $\square$
\dotline
\textbf{Dem: } Considere os conjuntos $S=\{p \inN ; mp = np \implies m = n
, \quad m,n \inN\}$ e \newline $T = \{m \inN ; mp=np\implies m=n,\quad n\inN, p\in S\}$
. Por construção, segue que $T \subset S$. Ora, $1 \in T$, já que, da unicida
de da identidade no produto e da proposição 1 do item c, $1\cdot p = np \implies n = 1$
. Vejamos que $s(T) \subset T$, supondo que $m \in T$ e escolhendo $n \inN$ e $
p \in S$ de tal forma que $s(m)p = np$, queremos mostrar que $s(m) = n$, sendo assim,
\begin{ceqnalign*}
  s(m) p &= np,\\
  p\cdot s(m) &= pn, \quad &&\text{(Comutatividade do produto em $\bN$)}\\
  pm + p &= pn. \quad &&\text{(Definição do produto em $\bN$)} \tag{(1)}
\end{ceqnalign*}

Note que, se $n = 1$, $p = p + pm$, o que por $(\star)$, teremos um absurdo. Portan
to $n \neq 1$, ou seja, existe $\omega \inN$ tal que $n = s(\omega)$. Sendo assim
, (1) pode ser reescrito da seguinte forma:
\begin{ceqnalign*}
  pm + p &= pn = p\cdot s(\omega),\\
  pm + p &= p\omega + p, \quad &&\text{(Definição do produto em $\bN$)}\\
  pm &= p\omega, \quad &&\text{(Lei do cancelamento da soma em $\bN$)}\\
  mp &= \omega p, \quad &&\text{(Comutatividade do produto em $\bN$)} \\
  m &= \omega, \quad &&\text{(Já que $p \in S$ e $m \in T$)}\\
  s(m) &= s(\omega), \quad && \text{(Injeção da função sucessor)}\\
  s(m) &= n. \quad &&\text{(Já que $n = s(\omega)$)}
\end{ceqnalign*}
Dito isto, já que como $1 \in T$ e $s(T) \subset T$, segue do Princípio da In
dução que $T = \bN$, mais ainda, $T \subset S \subset \bN$, mas como $T = \bN$
, segue que $S= \bN$, como queríamos. $\square$
%</ex1e>


%<*ex1f>
Demonstrado como proposição em (1e).
%</ex1f>


%<*ex2a>
\textbf{Dem:} Da hipótese, temos que $n < p$, isto é, $p = n + k$ para algum $k \inN$
. Ademais, $m \le n \equiv (n = m) \lor (n = m + \omega)$ para algum $\omega \inN$
. Com isto, suponhamos que $n = m$, portanto $p = n + k = m +k$, isto é $m < p$
; Agora, suponhamos que $n = m + \omega$,
\begin{ceqnalign*}
  p &= n + k, \\
  &= (m + \omega) + k, \\
  &= m + (\omega + k) = m + k_1, \quad k_1 \inN
\end{ceqnalign*}
ou seja, $m < p$. Já que $(n = m) \land (n<p) \implies m < p$ e $(m < n) \land (n<p) 
\implies m < p$, segue que $(m \le n) \land (n < p) \implies m < p$. $\square$ 
%</ex2a>


%<*ex2b>
\textbf{Dem: } Da hipótese, $m< n$, isto é, $n = m + k$ para algum $k \inN$, e
, $n \le p \equiv (p = n) \lor (p = n+k_1)$ para algum $k_1 \inN$. Supondo $p = n$,
segue que $m + k = n = p$, isto é, $m < p$; Agora, supondo $p = n + k_1$,
\begin{ceqnalign*}
  p &= n + k_1,\\
  &= (m+k) + k_1,\\
  &= m + (k + k_1) = m + k_2, \quad k_2 \inN
\end{ceqnalign*}
ou seja, $m < p$. Semelhante ao item anterior, segue que $(m < n) \land (n \le p) 
\implies m < p$. $\square$
%</ex2b>


%<*ex2c>
Dos itens (a) e (b), sabemos que:
\begin{itemize}
  \item $(m<n) \lor (n=p) \implies m < p$;

  \item $(m<n) \lor (n<p) \implies m < p$;

  \item $(m=n) \lor (n<p) \implies m < p$.
\end{itemize}
%</ex2c>
