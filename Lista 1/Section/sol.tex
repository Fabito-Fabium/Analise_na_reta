%<*ex1a>
Antes, considere a seguinte proposição:
\dotline
\textbf{Proposição:} Seja $m \inN$, então $m+1 = 1+ m$.\\
\textbf{Demonstração da proposição:} Considere $S \subset \bN$ tal que, $S = \{m \inN 
; m+1=1+m\}$. Note que $1 \in S$, já que $(1) + 1 = 2 = 1 + (1)$; Vejamos agora que $
s(S) \subset S$. Seja $m \in S$, então:
\begin{ceqnalign*}
  s(m)+1 &= (m+1) + 1, \quad &&\text{(Definição da função sucessor)}\\
  &= (1+m) + 1, \quad &&\text{(Visto que $m \in S$)} \\
  &= 1 + (m + 1), \quad &&\text{(Associatividade da soma em $\bN$)}\\
  &= 1 + s(m). \quad &&\text{(Definição da função sucessor)}
\end{ceqnalign*}

Deste modo, já que $1 \in S$ e $s(S) \in S$, pelo Princípio da Indução
, temos que $S = \bN$, como queríamos. $\square$
\dotline
\textbf{Dem:} Considere $S \subset \bN$ tal que $S = \{m \inN; m+n = n+m, n\inN\}$
. Da proposição anterior, temos que $1 \in S$, visto que $1 + n = n + 1$. Vejamos que $
s(S) \subset S$. Tomando $m \in S$,\\
\begin{ceqnalign*}
  s(m)+n &= (m+1) + n, \quad &&\text{(Definição da função sucessor)}\\
  &= m + (1+n), \quad &&\text{(Associatividade da soma em $\bN$)} \\
  &= m + (n+1), \quad &&\text{(Proposição anterior)}\\
  &= (m+n) + 1, \quad &&\text{(Associatividade da soma em $\bN$)}\\
  &= (n + m) + 1, \quad &&\text{(Visto que $m \in S$)}\\
  &= n + (m+1), \quad &&\text{(Associatividade da soma em $\bN$)}\\
  &= n + s(m). \quad&&\text{(Definição da função sucessor)}
\end{ceqnalign*}

Visto que $1 \in S$ e $s(S) \subset S$, pelo Princípio da Indução, temos $S=\bN$
, como queríamos. $\square$
%</ex1a>

%<*ex1b>
\textbf{Dem: } Seja $S \subset \bN$ tal que $S=\{p \inN ; (m+p=n+p) \implies m=n, \quad 
m,n \in \bN\}$. Ora, $1 \in S$, já que, se $m+1 = n+1$, então $s(m) = s(n)$, e
, como $s$ é injetora por definição, então $m=n$. Vejamos que $s(S) \subset S$. Toman
do $p \in S$ e $m+s(p) = n + s(p)$,
\begin{ceqnalign*}
  m+s(p)&=n+s(p),\\
  m+(p+1) &=n+(p+1), \quad &&\text{(Definição da função sucessor)}\\
  (m+p) + 1 &= (n+p) + 1, \quad &&\text{(Associatividade da soma em $bN$)}\\
  s(m+p)&=s(n+p), \quad &&\text{(Definição da função sucessor)}\\
  m+p &=n+p, \quad &&\text{(Injetividade da função sucessor)}\\
  m &= n. \quad &&\text{(Já que $p \in S$)}
\end{ceqnalign*}

Deste modo, como $1 \in S$ e $s(S) \subset S$, pelo Princípio da Indução, temos $S 
= \bN$. $\square$
%</ex1b>

%<*ex1c>
Antes, considere as seguintes proposições:
\dotline
\textbf{Proposição 1:} Seja $n \inN$, $n\cdot 1 = 1 \cdot n$.\\
\textbf{Demonstração da proposição:} Seja $T \subset \bN$ tal que $T = \{n\inN; n\cdot 1
=1\cdot n\}$, queremos mostrar que $T = \bN$. Ora, $1 \in T$, já que pela defini
ção do produto em $\bN$, $(1)\cdot 1 = 1 = 1 \cdot(1)$. Vejamos que $s(T) \subset T$
. Tomando $n \in T$, temos que:\\
\begin{ceqnalign*}
  s(n)\cdot 1 &= s(n), \quad&&\text{(Definição do produto em $\bN$)}\\
  &=n + 1, \quad&&\text{(Definição da função sucessor)}\\
  &=1\cdot n + 1, \quad&&\text{(Da definição do produto e sabendo que $n \in T$
  , $n = n\cdot 1 = 1 \cdot n$)}\\
  &= 1\cdot s(n). \quad &&\text{(Definição do produto em $\bN$, $m\cdot s(n) = mn+m$)}
\end{ceqnalign*}
Portanto, como $1 \in T$ e $s(T) \subset T$, segue do Princípio da Indução que $T=\bN$
. $\square$
\dotline
\textbf{Proposição 2:} (Distributiva comutada) Para cada $m, n, p \inN$, temos que 
\\
$$(n+p)m = nm+pm.$$
\textbf{Demonstração da proposição:} Considere $R=\{m \inN; (n+p)m = nm + pm, 
\quad n,p\inN\}$. Note que $1 \in R$, visto que, da definição de produto, $
(n+p)\cdot 1 = n + p = n\cdot 1 + p\cdot 1$. Vejamos que $s(R) \subset R$. Assumindo $
m \in R$, temos que:
\begin{ceqnalign*}
  (n+p)s(m) &= (n+p)m + (n+p), \quad &&\text{(Definição do produto em $\bN$)}\\
  &= (nm + pm) + (n+p), \quad &&\text{(Visto que $m \in R$, $(n+p)m = nm + pm$)}\\
  &= nm + (pm + (p+n)), \quad &&\text{(Associatividade e comutatividade da soma em $
  \bN$)}\\
  &= nm + (n + (pm + p)), \quad &&\text{(Associatividade e comutatividade da soma em $
  \bN$)}\\
  &= (nm + n) + (pm + p), \quad &&\text{(Associatividade da soma em $\bN$)}\\
  &= n\cdot s(m) + p \cdot s(m). \quad &&\text{(Definição do produto em $\bN$)}
\end{ceqnalign*}
Sendo assim, já que $1 \in R$ e $s(R) \subset R$, segue do Princípio da Indu
ção que $R = \bN$. $\square$
\dotline
\textbf{Dem: } Seja $S = \{n \inN | m\cdot n = n \cdot m, \quad m \inN\}$. Veja
mos que $S = \bN$. 
%</ex1c>


%<*ex1d>
%</ex1d>


%<*ex1e>
%</ex1e>


%<*ex1f>
%</ex1f>
