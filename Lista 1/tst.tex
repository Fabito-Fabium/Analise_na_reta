\usepackage[hyperlink]{qrcode}
\usepackage{environ}
\usepackage{tabularray}
\usepackage[fleqn]{nccmath}
\usepackage{bm}
\relpenalty=9999
\binoppenalty=9999
\NewEnviron{ceqnalign*}
    {\vspace{3mm}
    \begin{ceqn}
        \begin{align*}
            \BODY
        \end{align*}
\end{ceqn}}
\NewEnviron{ceqnalign}[1]
{\vspace{3mm}
	\begin{ceqn}
		\begin{align*}
			\BODY \tag{#1}
		\end{align*}
\end{ceqn}}
\usepackage{catchfilebetweentags}
\DeclareMathOperator{\Ima}{Im}
\newcommand{\Laplace}{\mathscr{L}}
\newcommand{\bN}{\mathbb N}
\newcommand{\inN}{\in \bN}
\newcommand{\dotline}{\\.\dotfill.\\}
\newcommand{\sqn}[1]{(#1_n)_{n \in \bN}}
\newcommand{\fmm}[3]{\{#1_#2\}_{#2 \in #3}}
\newcommand{\seq}[3]{(#1_#2)_{#2 \in #3}}
\newcommand{\tozero}[1]{#1 \rightarrow 0}
\newcommand{\toinf}[1]{#1 \rightarrow \infty}
\newcommand{\limsq}[3]{\lim_{#1 \rightarrow #2} #3_#1}
\newcommand{\limif}[2]{\lim_{#1 \rightarrow \infty} #2}
\newcommand{\limzr}[2]{\lim_{#1 \rightarrow 0} #2}
\newcommand{\bC}{\mathbb C}
\newcommand{\bZ}{\mathbb Z}
\newcommand{\bR}{\mathbb R}
\newcommand{\bQ}{\mathbb Q}
\newcommand{\itr}[1]{\text{int}(#1)}
\newcommand{\parth}[1]{\left(#1\right)}
\newcommand{\chavs}[1]{\left\{#1\right\}}
\newcommand{\mdl}[1]{\left|#1\right|}
\newcommand{\xssy}[2]{\left(#1 - #2, #1 + #2\right)}
\newcommand{\floor}[1]{\left\lfloor#1\right\rfloor}
\newcommand{\ceil}[1]{\left\lceil#1\right\rceil}
\newcommand{\asbeps}[2]{|#1 - #2| < \epsilon}
\newcommand{\iffc}[1]{\overset{#1}{\iff}}
\newcommand{\impliesc}[1]{\overset{#1}{\implies}}
\newcommand{\eqc}[1]{\overset{#1}{=}}
\newcommand{\sen}{\mathrm{sen}}
%% Início do documento
\begin{document}
\begin{enumerate}[wide, labelwidth=!, labelindent=0pt]
  \item \textbf{Como vimos, existe uma função injetora $\dfunc s \bN \bN$, tal que $s(n) = n+1$, em que 
    $s(n)$ chama-se \textit{sucessor} de $n$. A partir disso, podemos definir a operação \textit{soma} em 
    $\bN$, que satisfaz a seguinte lei de recursão: }\\
  \begin{itemize}
    \item \textbf{Para todo $n \inN$, temos que $n+1 = s(n)$;}
    \item \textbf{Para cada $m, n \inN$, temos que $n + s(m) = s(n+m)$.}
  \end{itemize}
  \vspace{3mm}
  \textbf{Também podemos definir a operação de produto em $\bN$, que satisfaz a seguinte lei de recursão:}\\
  \begin{itemize}
    \item \textbf{Para todo $n\inN$, temos que $n\cdot 1 = n$;}
    \item \textbf{Para cada $m, n \inN$, temos que $n\cdot s(m) = nm+n$.}
  \end{itemize}
    \vspace{3mm}
    \textbf{Mostre que valem as seguintes propriedades:}
  \begin{enumerate}[label=\alph*)]
      \item \textbf{(Comutatividade da soma) Para cada $m, n \inN$, temos que $m+n = n+m$.}\\
        
        
        
Antes, considere a seguinte proposição:
\dotline
\textbf{Proposição:} Seja $m \inN$, então $m+1 = 1+ m$.\\
\textbf{Demonstração da proposição:} Considere $S \subset \bN$ tal que, $S = \{m \inN 
; m+1=1+m\}$. Note que $1 \in S$, já que $(1) + 1 = 2 = 1 + (1)$; Vejamos agora que $
s(S) \subset S$. Seja $m \in S$, então:
\begin{ceqnalign*}
  s(m)+1 &= (m+1) + 1, \quad &&\text{(Definição da função sucessor)}\\
  &= (1+m) + 1, \quad &&\text{(Visto que $m \in S$)} \\
  &= 1 + (m + 1), \quad &&\text{(Associatividade da soma em $\bN$)}\\
  &= 1 + s(m). \quad &&\text{(Definição da função sucessor)}
\end{ceqnalign*}

Deste modo, já que $1 \in S$ e $s(S) \in S$, pelo Princípio da Indução, temos que $S = \bN$, como queríamos. $\square$
\dotline
\textbf{Dem:} Considere $S \subset \bN$ tal que $S = \{m \inN; m+n = n+m, n\inN\}$. Da proposição anterior, temos que $1 \in S$, visto que $1 + n = n + 1$. Vejamos que $
s(S) \subset S$. Tomando $m \in S$,\\
\begin{ceqnalign*}
  s(m)+n &= (m+1) + n, \quad &&\text{(Definição da função sucessor)}\\
  &= m + (1+n), \quad &&\text{(Associatividade da soma em $\bN$)} \\
  &= m + (n+1), \quad &&\text{(Proposição anterior)}\\
  &= (m+n) + 1, \quad &&\text{(Associatividade da soma em $\bN$)}\\
  &= (n + m) + 1, \quad &&\text{(Visto que $m \in S$)}\\
  &= n + (m+1), \quad &&\text{(Associatividade da soma em $\bN$)}\\
  &= n + s(m). \quad&&\text{(Definição da função sucessor)}
\end{ceqnalign*}

Visto que $1 \in S$ e $s(S) \subset S$, pelo Princípio da Indução, temos $S=\bN$, como queríamos. $\square$\\
        
        
      \item \textbf{(Lei do cancelamento da soma) Para cada $m, n, p \inN$, se $m+p = n+ p$, então $m=n$.}
        \ExecuteMetaData[Section/sol.tex]{ex1b}\\

      \item \textbf{(Comutatividade do produto) Para cada $m, n \inN$, temos que $mn = nm$.}\\
        \ExecuteMetaData[Section/sol.tex]{ex1c}\\
      \item \textbf{(Distributividade) Para cada $m,n,p \inN$, temos que $m(n+p) = mn + mp$.}\\
        \ExecuteMetaData[Section/sol.tex]{ex1d}\\
      
      \item \textbf{(Lei do cancelamento do produto) Para cada $m,n,p \inN$, se $mp=np$, então m=n.}
        \ExecuteMetaData[Section/sol.tex]{ex1e}\\
      
      \item \textbf{(Unicidade da identidade do produto) Para todo $n \inN$, se $mn = n$, então $m = 1$}\\
        \ExecuteMetaData[Section/sol.tex]{ex1f}\\
  \end{enumerate} 
	  \vspace{3mm}

  \item \textbf{Sejam $m, n, p \inN$. Mostre que}

  \begin{enumerate}[label=\alph*)]
    \item \textbf{Se $m \le n$ e $n < p$, então $m < p$.}\\
    \ExecuteMetaData[Section/sol.tex]{ex2a}\\
    \item \textbf{Se $m < n$ e $n \le p$, então $m < p$.}\\
    \ExecuteMetaData[Section/sol.tex]{ex2b}\\
    \item \textbf{Se $m \le n$ e $n \le p$, então $m \le p$.}\\
    \ExecuteMetaData[Section/sol.tex]{ex2c}\\
  \end{enumerate}
  
\end{enumerate}


\end{document}
