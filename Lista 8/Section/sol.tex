%<*ex1>
\textbf{Dem:} Vejamos que $A = (a, b)$ é um conjunto aberto. Note que para todo $x \in A$, $x \in \bR$ e $a<x<b$, (1.0). Sejam $\overline{a} = x - a$ e $\overline{b} = b - x$. Como $x>a$, segue que $\overline{a} = x-a > 0$, ou seja $\overline{a} > 0$, analogamente temos $\overline{b} > 0$. Com isto, considere $r = \min(\overline{a}, \overline{b})$, vejamos então, que para todo $x \in A$, $\parth{x - \frac{r}{2}, x + \frac{r}{2}} \subset A$. Ora, como $r = \min(\overline{a}, \overline{b})$, segue que $r \leq \overline{a}$ e $r \leq \overline{b}$. Ao considerar $r \leq \overline{a}$, isto é, $-r \geq -\overline{a}$, podemos verificar que,

\begin{ceqnalign*}
	x - \frac{r}{2} \geq x - \frac{\overline{a}}{2} = x - \frac{x - a}{2} = \frac{x + a}{2},
\end{ceqnalign*}

e, como $x > a$, segue que $x - \frac{r}{2} \geq \frac{x+a}{2} > \frac{a+a}{2} = a$, isto é, $x - \frac{r}{2} > a$. Agora, considere $r \leq \overline{b}$, ou seja,

\begin{ceqnalign*}
	x + \frac{r}{2} \leq x + \frac{\overline{b}}{2} = x + \frac{b - x}{2} = \frac{b+x}{2},
\end{ceqnalign*}

mas como $x<b$, segue que $x+\frac{r}{2} < \frac{b+b}{2} = b$, portanto $x+\frac{r}{2} < b$. Ademais, visto que $\overline{b}>0$ e $\overline{a}>0$, temos que $r = \min(\overline{a}, \overline{b}) > 0$, ou seja, $-r < r$. Isto posto, temos que,
\begin{ceqnalign}{1.1}
	a<x-\frac{r}{2} < x + \frac{r}{2} < b
\end{ceqnalign}
e como para todo $x \in A$ temos (1.0), então (1.1) pode ser expressa por $\parth{x - \frac{r}{2}, x + \frac{r}{2}} \subset A$. Com isso, podemos concluir que $A \subset \itr{A}$. Mais ainda, pela definição de pontos interiores, temos que $\itr{A} \subset A$, portanto $A = \itr{A}$ que, pela definição de abertos, segue que $A = (a, b)$ é um aberto, como queríamos.\\

Vejamos que $B = (-\infty, b)$ é um conjunto aberto. Análogo ao caso anterior, considere $\overline{b} = b - x > 0$ tal que $x\in B$. Como $x + \frac{\overline{b}}{2} = \frac{b+x}{2} < \frac{b+b}{2} = b$, e $-\overline{b} < \overline{b}$, então
\begin{ceqnalign}{1.2}
	x-\frac{\overline{b}}{2} < x + \frac{\overline{b}}{2}<b 
\end{ceqnalign}
e, visto que $x \in B$ se, e somente se, $x \in \bR$ e $x < b$, segue que $\parth{x-\frac{\overline{b}}{2},x+\frac{\overline{b}}{2}} \subset B$, (1.3). Mais ainda, como para todo $x \in B$ temos (1.2), e por conseguinte (1.3), então $B \subset \itr B$ e, da definição de pontos interiores, $\itr B \subset B$, ou seja $B = \itr B$. Em vista disso, pela definição de abertos, segue que $B$ é aberto, como queríamos.\\

Vejamos agora que $C = (a, \infty)$ é um conjunto aberto. Como visto anteriormente, sejam $x \in A$ e $\overline{a} = x - a > 0$. Visto que $x - \frac{\overline{a}}{2} = \frac{x + a}{2} > a$ e $-\overline{a} < \overline{a}$, então,
\begin{ceqnalign}{1.4}
	a < x-\frac{\overline{a}}{2} < x + \frac{\overline{a}}{2},
\end{ceqnalign}
Portanto $\parth{x-\frac{\overline{a}}{2}, x + \frac{\overline{a}}{2}} \subset C$, (1.5). Já que para todo $x \in C$ temos (1.5), segue que $C \subset \itr C$, e da definição de pontos interiores, $\itr C \subset C$, ou seja $C = \itr C$, sendo assim, $C$ é aberto, como queríamos. $\square$
%</ex1>


%<*ex2>
\textbf{Dem:}  Suponhamos que $a\in A$ é um ponto de acumulação de $A$, ou seja, existe uma sequência  $\sqn{x}$ tal que $x_n \in A \backslash \{a\}$ e $\lim\limits_{n \rightarrow \infty} x_n = a$. Como $\lim\limits_{\toinf{n}} x_n = a$, para todo $\epsilon > 0$ existe $n_0 \in \bN$ tal que, para $n \in \bN$, temos $|x_n - a| < \epsilon$ com $n \ge n_0$, ou seja

\begin{ceqnalign}{2.1}
	|x_n - a| < \epsilon \iff -\epsilon < x_n - a < \epsilon \implies a - \epsilon < x_n < a + \epsilon
\end{ceqnalign}

e, como (2.1) vale para todo $\epsilon > 0$, é conveniente  escolher um $\epsilon$ suficiente pequeno, tal que  $(a - \epsilon, a + \epsilon)$ é uma vizinhança de $a$ de raio $\epsilon$ centrada em $a$ que esteja contida em $A$ e que denotaremos por $V_\epsilon(a)$. Com isto, como $\epsilon > 0$, existe um ponto $a - \frac{\epsilon}{2} \in (a - \epsilon, a + \epsilon) = V_\epsilon(a) = V_\epsilon(a) \cap A$ que é diferente de $a$, ou seja, em particular, $a - \frac{\epsilon}{2} \in (V_\epsilon(a) \cap A) \backslash \{a\} = V_\epsilon(a) \cap (A \backslash \{a\})$. Com isto, temos o que queríamos.\\
Por outro lado, suponhamos que toda vizinhança $V$ de $a$ contém um ponto de $A\backslash\{a\}$, (2.2). Ora, como (2.2) vale para qualquer vizinhança $V(a)$, então, seja $V(a) = (\alpha, \beta)$, tal que $\alpha, \beta \in \bR$ e $\alpha < \beta$. Da hipótese, sabemos que $V(a) \cap (A \backslash \{a\}) \neq \emptyset$, portanto, podemos construir uma sequência $(x_n)_{n \in \bN}$ em $V(a)$ tal que $\lim\limits_{\toinf{n}} x_n = a$. Com isto, sejam $\overline{\alpha} = a - \alpha$ e $\overline{\beta} = \beta - a$, e $r = \min (\overline{\alpha}, \overline{\beta})$, e, considere a sequência $(x_n)_{n \in \bN}$ tal que $x_n = a + \frac{r}{n+1}$ . Note que $\lim\limits_{\toinf{n}}x_n = a$, mais ainda, por construção, para todo $n \in \bN$, $x_n \in V(a)\backslash\{a\}$, portanto,  $a$ é um ponto de acumulação, como queríamos. $\square$
%</ex2>

%<*ex3>
Antes da demonstração, devemos relembrar da seguinte proposição e de um teorema demonstrados em aula.
\newline
\textbf{Proposição:} Seja $A$ um conjunto contável. Se $B \subset A$, então $B$ é contável.\\
A proposição acima é logicamente equivalente a seguinte afirmação:\\
\textbf{Proposição:} Sejam dois conjuntos $A$ e $B$ tais que $B \subset A$. Se $A$ é contável então $B$ é contável.\\
Mais ainda, a contrapositiva da reescrita proposta nos diz que:\\
\textbf{Lema:} Sejam dois conjuntos $A$ e $B$ tais que  $B \subset A$. Se $B$ não é contável, então $A$ não é contável.\\
\newline
\textbf{Teorema:} Todo intervalo $\mathcal I$, não-degenerado é não-enumerável.\\
\newline
\textbf{Demonstração do exercício proposto:} Seja um conjunto qualquer  $A \subset \bR$ enumerável, ou seja, contável, e, afim de contradição, suponha que $\itr A \neq \emptyset$, isto é, existe pelo menos um $a \in \itr A \subset A$.  Note que, se $A=\emptyset$, temos uma contradição; Visto que há pelo menos um $a \in \itr A$, da definição de ponto interior, existe $\epsilon > 0$ tal que $\xssy{a}{\epsilon} \subset A$. Como $\xssy{a}{\epsilon}$ é um intervalo não-degenerado, do teorema citado, segue que $\xssy{a}{\epsilon}$ é não-enumerável, portanto $\xssy{a}{\epsilon}$ não é contável. Daqui, visto que $\itr A \subset A$ e $\itr A$ não é contável, segue do lema acima que $A$ não é contável, o que contradiz a hipótese. Portanto $\itr A = \emptyset$, como queríamos. $\square$\\

\textbf{Exemplos de conjuntos com interior vazio:}
\begin{enumerate}
	\item Qualquer conjunto unitário, isto é, $A=\{a\}$ onde $a \in \bR$, tem interior vazio, visto que para todo $\epsilon>0$, $\xssy{a}{\epsilon} \not \subset A$. \textbf{Exemplo:} $A_0 = \{0\}, A_1 = \{1\}, A_{-1} = \{-1\}$ tem interior vazio.
	\item Qualquer conjunto finito contido nos reais, isto é,  $A=\{x_0, \dots, x_n\}$ onde $n \in \bN$ e $x_k \in \bR$ para todo $k \in \bZ \cap [0, n]$ e $x_i < x_j$ quando $i < j$, tem interior vazio.\\
	 Demonstração alternativa para este caso em específico: Considere $d = \min\{|x_i - x_{i+1}|; i = 0, 1, \dots, n - 1\}$, para cada $d> \epsilon_0 > 0$ e $k \in \bZ \cap [0, n]$, não existe $a \in A$ tal que $a\neq x_k$ e $a \in (x_k - \epsilon_0, x_k + \epsilon_0)$, já que, caso contrário, violaríamos a definição de $d$; ou seja, $(x_k - \epsilon_0, x_k + \epsilon_0) \not \subset A$ . Note que, para todo $\epsilon > 0$ existe pelo menos um $\epsilon_0$ como descrito anteriormente, tal que $I_{0, k} = \xssy{x_k}{\epsilon_0} \subset \xssy{x_k}{\epsilon} = I_{1, k}$ com $k \in \bZ \cap [0, n]$, portanto, como $I_{0, k} \not\subset A$, $I_{0, k} \subset I_{1, k}$ e $I_{0, k} \cap I_{1, k} \neq \emptyset$, segue que $I_{1, k} \not \subset A$, em outras palavras, para cada $k \in \bZ \cap [0, n]$ e $\epsilon>0$, temos que $\xssy{x_k}{\epsilon} \not \subset A$, isto é, $\itr A = \emptyset$. \textbf{Exemplo:} $A_0 = \{0, 1\}, A_1 = \{0, 1, 2\}, A_3 = \{0, 1, 2, 3\}$ tem interior vazio.
\end{enumerate}
%</ex3>

%<*ex4a>
\textbf{Dem:} Da definição de interior, para todo $x \in (\itr(A) \cup \itr(B))$, existe $\epsilon>0$ tal que $\xssy{x}{\epsilon} \subset (A \cup B)$
%</ex4a>