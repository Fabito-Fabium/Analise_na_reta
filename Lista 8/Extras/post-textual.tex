%%%% Elementos pós-textuais
%%
%% Glossário, apêndices e anexos.

%% Glossário (inserir itens em ordem alfabética)
\begin{Glossary}%[\bfseries]%% Estilo de fonte do termo
\item[Bib\LaTeX] reimplementação completa das facilidades bibliográficas fornecidas pelo \LaTeX.
\item[Bib\TeX] aplicativo de gerenciamento de referências para a formatação de listas de referências no \LaTeX.
\item[GEDWeb] sistema desenvolvido para gerenciar grandes acervos de normas e informações técnicas.
\item[ID Lattes] sequência de 16 dígitos que permite que qualquer pessoa encontre o Currículo Lattes de um{(a)} pesquisador{(a)}.
\item[\LaTeX] conjunto de macros para o processador de textos \TeX, utilizado amplamente para a produção de textos matemáticos e científicos devido à sua alta qualidade tipográfica.
\item[ORCID] código alfanumérico não proprietário para identificar exclusivamente cientistas, outros autores acadêmicos e contribuidores (\ENLang*{Open Researcher and Contributor ID} ou ID Aberto de Pesquisador e Contribuidor).
\item[\TeX] sistema de tipografia criado por Donald E. Knuth.
\item[\UTFPR-Article] modelo \LaTeX\ (não oficial) que permite a produção de artigos da Universidade Tecnológica Federal do Paraná (UTFPR), inclusive para eventos.
\end{Glossary}

%% Apêndices
\begin{Appendix}

\section{Título de Apêndice}%
\label{sect:apx-a1}

Exemplo de apêndice (\Cref{sect:apx-a1}) em uma seção de \nameref{sect:appendix}.

\subsection{Título de Seção Secundária de Apêndice}%
\label{ssect:apx-a2}

Exemplo de seção secundária de apêndice (\Cref{ssect:apx-a2}).

\subsubsection{Título de Seção Terciária de Apêndice}%
\label{sssect:apx-a3}

Exemplo de seção terciária de apêndice (\Cref{sssect:apx-a3}).

\paragraph{Título de seção quaternária de Apêndice}%
\label{prgh:apx-a4}

Exemplo de seção quaternária de apêndice (\Cref{prgh:apx-a4}).

\subparagraph{Título de seção quinária de Apêndice}%
\label{sprgh:apx-a5}

Exemplo de seção quinária de apêndice (\Cref{sprgh:apx-a5}).

\end{Appendix}

%% Anexos
\begin{Annex}

\section{Título de Anexo}%
\label{sect:anx-a1}

Exemplo de anexo (\Cref{sect:anx-a1}) em uma seção de \nameref{sect:annex}.

\subsection{Título de Seção Secundária de Anexo}%
\label{ssect:anx-a2}

Exemplo de seção secundária de anexo (\Cref{ssect:anx-a2}).

\subsubsection{Título de Seção Terciária de Anexo}%
\label{sssect:anx-a3}

Exemplo de seção terciária de anexo (\Cref{sssect:anx-a3}).

\paragraph{Título de seção quaternária de Anexo}%
\label{prgh:anx-a4}

Exemplo de seção quaternária de anexo (\Cref{prgh:anx-a4}).

\subparagraph{Título de seção quinária de Anexo}%
\label{sprgh:anx-a5}

Exemplo de seção quinária de anexo (\Cref{sprgh:anx-a5}).

\end{Annex}
